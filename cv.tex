% FortySecondsCV LaTeX template
% Copyright © 2019-2020 René Wirnata <rene.wirnata@pandascience.net>
% Licensed under the 3-Clause BSD License. See LICENSE file for details.
%
% Please visit https://github.com/PandaScience/FortySecondsCV for the most
% recent version! For bugs or feature requests, please open a new issue on
% github.
%
% Contributors
% ------------
% * ifokkema
% * Bertbk
% * Hespe
% * esben
%
% Attributions
% ------------
% * fortysecondscv is based on the twentysecondcv class by Carmine Spagnuolo
%   (cspagnuolo@unisa.it), released under the MIT license and available under
%   https://github.com/spagnuolocarmine/TwentySecondsCurriculumVitae-LaTex
% * further attributions are indicated immediately before corresponding code


%-------------------------------------------------------------------------------
%                             ADDITIONAL PACKAGES
%-------------------------------------------------------------------------------
\documentclass[
	a4paper,
	% showframes,
	% vline=2.2em,
	% maincolor=cvgreen,
	% sidecolor=gray!50,
	% sectioncolor=red,
	% subsectioncolor=orange,
	% itemtextcolor=black!80,
	% sidebarwidth=0.4\paperwidth,
	% topbottommargin=0.03\paperheight,
	% leftrightmargin=20pt,
	% profilepicsize=4.5cm,
	% profilepicborderwidth=3.5pt,
	% profilepicstyle=profilecircle,
	% profilepiczoom=1.0,
	% profilepicxshift=0mm,
	% profilepicyshift=0mm,
	% profilepicrounding=1.0cm,
	% logowidth=4.5cm,
	% logospace=5mm,
	% logoposition=before,
	% sidebarplacement=right,
]{fortysecondscv}

% fine tune line spacing
% \usepackage{setspace}
% \setstretch{1.1}

% improve word spacing and hyphenation
\usepackage{microtype}
\usepackage{ragged2e}

% uncomment in case you don't want any hyphenation
% \usepackage[none]{hyphenat}

% take care of proper font encoding
\ifxetexorluatex
	\usepackage{fontspec}
	\defaultfontfeatures{Ligatures=TeX}
	% \newfontfamily\headingfont[Path=fonts/]{segoeuib.ttf} % use local font
\else
	\usepackage[utf8]{inputenc}
	\usepackage[T1]{fontenc}
\fi

% use a sans serif font as default
\usepackage[sfdefault]{ClearSans}
% \usepackage[sfdefault]{noto}

% multi-language CV XeLaTeX and polyglossia (should also work with LuaLaTeX)
% NOTE: breaks \pointskill, \membership and some spacings
% \ifxetexorluatex
% 	\usepackage{polyglossia}
% 	\newfontfamily\arabicfontsf[Script=Arabic,Scale=1.5]{Amiri}
% 	\newfontfamily\englishfontsf{Clear Sans}
% 	\setmainfont{Amiri}
% 	\setdefaultlanguage{arabic}
% 	\setotherlanguage{english}
% \fi

% enable mathematical syntax for some symbols like \varnothing
\usepackage{amssymb}

% bubble diagram configuration
\usepackage{smartdiagram}
\smartdiagramset{
	% default font size is \large, so adjust to harmonize with sidebar layout
	bubble center node font = \footnotesize,
	bubble node font = \footnotesize,
	% default: 4cm/2.5cm; make minimum diameter relative to sidebar size
	bubble center node size = 0.4\sidebartextwidth,
	bubble node size = 0.25\sidebartextwidth,
	distance center/other bubbles = 1.5em,
	% set center bubble color
	bubble center node color = maincolor!70,
	% define the list of colors usable in the diagram
	set color list = {maincolor!10, maincolor!40,
	maincolor!20, maincolor!60, maincolor!35},
	% sets the opacity at which the bubbles are shown
	bubble fill opacity = 0.8,
}


%-------------------------------------------------------------------------------
%                            PERSONAL INFORMATION
%-------------------------------------------------------------------------------
%% mandatory information
% your name
\cvname{Romain\\Chevallier}
% job title/career
\cvjobtitle{Responsable en Ingénierie\\des Logiciels}

%% optional information
% profile picture
%\cvprofilepic{pics/profile.png}

% NOTE: ordering in sidebar will mimic the following order
% date of birth
\cvbirthday{08 Jan 1995}
% short address/location, use \newline if more than 1 line is required
\cvaddress{10, rue Robert Delaunay à DIJON}
% phone number
\cvphone{+33(0)6 84 94 36 04}
% email address
\cvmail{romain.chevallier21@gmail.com}
% any other custom entry
\cvcustomdata{\faFlag}{Français}
% personal website
\cvsite{https://www.linkedin.com/in/romain-chevallier-b7828a159/}
\cvcustomdata{\faGithub}
				{@RcDevRIL}

%-------------------------------------------------------------------------------
%                              SIDEBAR 1st PAGE
%-------------------------------------------------------------------------------
% add more profile sections to sidebar on first page
\addtofrontsidebar{
	% include gosquare national flags from https://github.com/gosquared/flags;
	% naming according to ISO 3166-1 alpha-2 country codes
	\graphicspath{{pics/flags/}}

	\profilesection{Langages}
		\skill{\flag{FR.png}}{Français - langue maternelle}
		\skill{\flag{GB.png}}{Anglais - courant}
		\skill{\flag{DE.png}}{Allemand - débutant}

	\profilesection{Informatique}
		{\textbf{Langages} :}
		\newline
			\skill{}{Java, Dart, Python, Javascript, HTML5, CSS, PHP, C/C++, SQL, Shell (UNIX)}
		{\textbf{Frameworks} :}
		\newline
			\skill{}{Apache Camel, Cucumber, Flutter, Spring, Spring Boot, React-native, Hibernate}
		{\textbf{Outils/Logiciels} :}
		\newline
			\skill{}{Jenkins, Git, Github, Azure DevOps, JBoss Fuse, PostgreSQL, Trello, Wix}
		{\textbf{Gestion de projet} :}
		\newline
			\skill{}{Azure DevOps Boards, Trello, Github Boards, Microsoft Project}

	\profilesection{Atouts}
		\skill{\faCompress}{\textbf{Adaptabilité}}
		\skill{\faCompress}{\textbf{Autonomie}}
		\skill{\faCompress}{\textbf{Curiosité}}

	\profilesection{Centres d'intérêts}
		\skill{}{\textbf{Sports ou loisirs} : Handball, Basket, E-sports}
		\skill{}{\textbf{Autres intérêts} : Voyages, Musique, Guitare, Nouvelles technologies, Aérospatial, Climat}
}

%-------------------------------------------------------------------------------
%                         TABLE ENTRIES RIGHT COLUMN
%-------------------------------------------------------------------------------
\begin{document}

\makefrontsidebar

\cvsection{Expériences}
\begin{cvtable}[3]
	\cvitem{2018 -- 2020}{Analyste Développeur Middleware en alternance}{ECONOCOM SAS}{\\- Service Middleware en lien direct avec les métiers (RH, Finance, ...)\\- Spécifications techniques et fonctionnelles\\- Implémentation de tests automatisés et interfaces\\- Focus DevOps\\- Approche Agile}
	\cvitem{2016 -- 2017}{Apprenti Ingénieur Qualité}{STMicroelectronics SAS}{\\- Supervision d'un parc de machines industrielles contrôlant la\\qualité des wafers de silicium produits\\- Rédaction de rapports de qualité des processus\\- Méthode LEAN}
	\cvitem{2011 -- 2016}{Jobs étudiant}{}{\\- Équipier polyvalent chez McDonald's\\- Livreur de pizza chez MrPizza et PizzaHut\\- Remplacements sur des postes d'agents d'accueil ou de caisse chez BNPParibas}
\end{cvtable}


\cvsection{Formation}
% sous-titre
% \cvsubsection{Post BAC}
\begin{cvtable}[1.5]
	\cvitem{2018 -- 2020}{Responsable en Ingénierie des Logiciels}{CESI Dijon-Quetigny}
		{Pendant cette formation j'ai eu l'occasion de rencontrer des professionnels du secteur lors d'un ensemble de modules permettant d'appréhender la gestion de projets de toutes tailles tout en restant très compétent sur l'aspect technique.}
	\cvitem{2017 -- 2018}{Cycle Ingénieur 1ère année}{ESIREM Dijon}
		{Cette année dans la filière Informatique et Électronique m'a permis de revoir quelques bases dans les matières scientifiques tout en confortant mon idée de vouloir travailler dans l'informatique logicielle.}
	\cvitem{2016 -- 2017}{Cycle Ingénieur en Alternance 1ère année}{ITII PACA}
		{Cette année en partenariat avec Polytech'Nice m'a permis de m'épanouir avec l'alternance dans une entreprise prestigieuse de l'électronique. (cf. Expériences)}
	\cvitem{2014 -- 2016}{Classe préparatoire intégrée}{Polytech'Nice - Sophia Antipolis}
		{Valider cette classe préparatoire m'a permis de gagner en rigourosité et d'approfondir mes connaissances scientifiques.}
	\cvitem{2013 -- 2014}{Classe Préparatoire PTSI}{Lycée Gustave Eiffel - Dijon}{}
	\cvitem{2010 -- 2013}{Baccalauréat Scientifique - Sciences de l'ingénieur}{Lycée Gustave Eiffel - Dijon}
		{- Informatique et Sciences du Numérique\\- Anglais Avancé (voyage en immersion aux États-Unis)}
\end{cvtable}

\cvsection{Compétences}
\begin{cvtable}
	\cvpubitem{}{Ma dernière expérience m'a permis d'appréhender l'utilisation professionnelle de langages, frameworks et outils populaires du secteur.\\De plus j'ai pu travailler avec une équipe anglophone de développeurs à distance.}
		{}{\textbf{Développement}}
	\cvpubitem{}{Lors de mes différentes expériences, au travers d'un apprentissage théorique et d'une mise en pratique en milieu professionnel, j'ai pu acquérir de bonnes bases dans la gestion de projet.}
		{}{\textbf{Gestion de projet}}
\end{cvtable}
% Put a date on the left and name on right
%\cvsignature
\end{document}
